\ifx\allfiles\undefined
\documentlecture[12pt, a4paper, oneside, UTF8]{ctexbook}  %  这一句是新增加的
\usepackage[dvipsnames]{xcolor}
\usepackage{amsmath}   % 数学公式
\usepackage{graphicx}
% \usetikzlibrary{arrows, calc, decorations.pathmorphing}
\allowdisplaybreaks % 允许公式跨页换行
\newcommand{\pa}{\partial}
\newcommand{\mT}{\raisebox{0.1ex}{$\scriptstyle -T$}} % 调整高度为 0.1ex
\newcommand{\lmT}{\raisebox{-0.85ex}{$\scriptstyle -T$}} % 调整高度为 -0.85ex
\newcommand{\mone}{\raisebox{0.1ex}{$\scriptstyle -1$}} % 调整高度为 0.1ex
\newcommand{\lmone}{\raisebox{-0.85ex}{$\scriptstyle -1$}} % 调整高度为 -0.85ex
\newcommand{\lsup}[1]{\raisebox{-0.1ex}{$\scriptstyle #1$}}
\newcommand{\llsup}[1]{\raisebox{-0.85ex}{$\scriptstyle #1$}}
\newcommand{\mathminus}{\!\!-\!\!} % 数学环境连字符
\newcommand{\vvec}[1]{\symbf{#1}}
\newcommand{\ten}[1]{\mathbb{#1}}
\newcommand{\red}[1]{\textcolor{red}{#1}}

\begin{document}
%\title{\Huge{\textbf{}}}
\author{作者:无名氏马}
\date{\today}
\maketitle                   % 在单独的标题页上生成一个标题

\thispagestyle{empty}        % 前言页面不使用页码
\begin{center}
    \Huge\textbf{前言}
\end{center}


\begin{flushright}
    \begin{tabular}{c}
        \today
    \end{tabular}
\end{flushright}

\newpage                      % 新的一页
\pagestyle{plain}             % 设置页眉和页脚的排版方式(plain:页眉是空的,页脚只包含一个居中的页码)
\setcounter{page}{1}          % 重新定义页码从第一页开始
\pagenumbering{Roman}         % 使用大写的罗马数字作为页码
\tableofcontents              % 生成目录

\newpage                      % 以下是正文
\pagestyle{plain}
\setcounter{page}{1}          % 使用阿拉伯数字作为页码
\pagenumbering{arabic}
% \setcounter{chapter}{-1}    % 设置 -1 可作为第零章绪论从第零章开始
 % 单独编译时,其实不用编译封面目录之类的,如需要不注释这句即可
\else
\fi
%  ↓↓↓↓↓↓↓↓↓↓↓↓↓↓↓↓↓↓↓↓↓↓↓↓↓↓↓↓ 正文部分
\chapter{数值计算}
\section{绪论}
\begin{definition}
    若\(A\)非奇异,称\(cond(A)=\|A\|\|A^{-1}\|\)为\(A\)的条件数。
    \(cond(A)\ge 1\),当\(A\)为正交矩阵时\(cong(A)=1\)。
\end{definition}
\begin{theorem}
    对于线性方程组\(Ax=b\)
    \begin{enumerate}
        \item 设\(A(x+\delta x)=b+\delta b\text{,}\delta x\text{收到}\delta b \text{的影响表示为}\)
        \[\frac{\|\delta x\|}{\|x\|}\le cond(A)\frac{\|\delta b\|}{\|b\|}\]
        \item 设\((A+\delta A)(x+\delta x)=b+\delta b\text{,}\delta x\text{收到}\delta A,\delta b \text{的影响表示为}\)
        \[\frac{\|\delta x\|}{\|x\|}\le \frac{cond(A)}{1-cond\frac{\|\delta A\|}{\|A\|}}\left(\frac{\|\delta A\|}{\|A\|}
        +\frac{\|\delta b\|}{\|b\|}\right)\]
    \end{enumerate}
    \begin{proof}
        \begin{gather*}
            (A+\delta A)(x+\delta x)=b+\delta b\\
            (A+\delta A)\delta x=b+\delta b -\cancelto{=b+\delta A\cdot x}{A+\delta A}\\
            \delta x=\underbrace{(A+\delta A)^{-1}}_{(I+\delta A A^{-1})^{-1}}(\delta b -\delta A\cdot x)\\
            \|\delta x\|=\|A^{-1}(I+\delta A A^{-1})^{-1}(\delta b-\delta A \cdot x)\|
            \\\le \|A^{-1}\|\|(I+\delta A A^{-1})^{-1}\|\|(\delta b-\delta A \cdot x)\|\\
            \le  \|A^{-1}\|\frac{1}{1-\|\delta A\|\| A^{-1}\|}\|(\underbrace{\delta b}_{partA}-\underbrace{\delta A \cdot x}_{partB})\|\\
            partB = \|A^{-1}\|\|A\|\frac{1}{1-\|\delta A\|\| A^{-1}\|}\frac{\|x\|\|\delta A\|}{\|A\|}\\
            partA=\|A^{-1}\|\frac{1}{1-\|\delta A\|\| A^{-1}\|}\frac{\|\delta b\|\|Ax\|}{\|B\|}
            \le \|A^{-1}\|\frac{1}{1-\|\delta A\|\| A^{-1}\|}\frac{\|\delta b\|\|A\|\|x\|}{\|B\|}\\
            \frac{\|\delta x\|}{\|x\|}\le \frac{cond(A)}{1-cond\frac{\|\delta A\|}{\|A\|}}\left(\frac{\|\delta A\|}{\|A\|}
        +\frac{\|\delta b\|}{\|b\|}\right)
        \end{gather*}
    \end{proof}
\end{theorem}
\begin{lemma}
    若\(A\in \R^{n\times n}\text{,且}\|A\|<1\),
    则\begin{enumerate}
        \item \(I-A\)非奇异
        \item \((I-A)^{-1}=\sum_{k=0}^{\infty}A^k\)
        \item \(\|(I-A)^{-1}\|\le \frac{1}{1-\|A\|}\)
    \end{enumerate}
    \begin{proof}
        \begin{enumerate}
            \item 
        若\(I-A\)奇异,则\(\exists x\neq0\text{,}s.t.(I-A)x=0\text{,即1为}A\text{的特征值,矛盾}\)
        \item 
        \(\sum_{k=0}^{\infty}A^k\cdot (I-A)=A^0=I\)
        \item         \(\|(I-A)^{-1}\|= \|\sum_{k=0}^{\infty}A^k\|\le \sum_{k=0}^{\infty}\|A^k\|
        \le \sum_{k=0}^{\infty}\|A\|^k=\frac{1}{1-\|A\|}\)
    \end{enumerate}
    \end{proof}
\end{lemma}
\begin{note}
在数值分析中,判断矩阵 $A$ 是否病态(即对微小扰动敏感)通常不直接计算 $A^{-1}$(条件数),而是依据以下经验准则:
\begin{enumerate}
    \item \textbf{行列式很大或很小(如某些行、列近似相关)} \\
    行列式 $\det(A)$ 的绝对值若远小于或远大于矩阵元素数量级的 $n$ 次方,可能预示病态。\\
    \textbf{解释:} 行列式绝对值过小通常意味着矩阵的行或列向量近似线性相关,导致其张成的“体积”很小,矩阵接近奇异。

    \item \textbf{元素间相差大数量级,且无规则} \\
    若矩阵元素在数值上跨越多个数量级(如 $10^{-6}$ 与 $10^3$ 共存),且没有特定的分布模式(如对角占优)。\\
    \textbf{解释:} 这种巨大的尺度差异会在数值计算(如消元法)中放大舍入误差,除非该矩阵具有特殊的、可补偿的结构。

    \item \textbf{主元消去过程中出现小主元} \\
    在进行高斯消元时,如果出现绝对值非常小的主元。\\
    \textbf{解释:} 小主元意味着矩阵接近奇异,在消元过程中需要用其去除其他大元素,这会极大地放大原始数据中存在的微小误差。

    \item \textbf{特征值相差大数量级} \\
    矩阵的特征值 $\lambda_i$ 的绝对值分布范围极广,即 $\max|\lambda_i| / \min|\lambda_i| \gg 1$。\\
    \textbf{解释:} 对于对称矩阵,这直接导致其条件数 $\kappa(A)$ 极大,意味着矩阵在不同特征向量方向上的伸缩程度差异巨大,从而对扰动非常敏感。
\end{enumerate}
\end{note}

\section{非线性方程求根}
\subsection{二分法(对分法)}
\mbox{}
二分法两种终止条件:
\begin{enumerate}
    \item  给定精度条件 \( |f(x)| < \varepsilon \)
    \textbf{不适用情况:}  
    当函数在远离根的区域存在\textbf{平坦区间}(即导数近似为零)时,即使当前点 \(x\) 离真实根 \(r\) 还很远,也可能满足 \( |f(x)| < \varepsilon \),导致算法提前终止,返回一个错误解。
    
    \item 区间长度条件 \( b - a < \varepsilon \)
    \textbf{不适用情况:}  
    当函数在根 \( r \) 附近\textbf{数值上呈现陡峭跳变}(由于浮点舍入误差导致类似符号函数的行为)时,可能在迭代过程中因符号判断错误而将真根排除在区间之外。此时区间继续缩小至 \( b-a < \varepsilon \),但区间内不含根,导致返回错误解。
\end{enumerate}

二分法缺点:
  \begin{enumerate}
    \item 只能求单个根,只能计算实根。
    \item 即使\(f(x)\text{在}[a,b]\text{上有根,也未必有}f(a)f(b)<0\)。
  \end{enumerate}  
\begin{note}
    常用\(sgn(f(a))\cdot sgn(f(x))> 0\text{代替}f(a)\cdot f(x)>0\)的判断,避免\(f(a)\cdot f(x)>0\)数值溢出。
\end{note}
\subsection{不动点迭代法}
\begin{theorem}
若 \(\varphi(x)\) 定义在 \([a, b]\) 上,如果 \(\varphi(x)\) 满足
\begin{enumerate}
    \item 当 \(x \in [a, b]\) 时有
    \(
    a \leq \varphi(x) \leq b
    \);
    \item \(\varphi(x)\) 在 \([a, b]\) 上可导,并且存在正数 \(L < 1\),使对任意的 \(x \in [a, b]\),
    \(
    |\varphi'(x)| \leq L
    \)。
\end{enumerate}
\begin{enumerate}
    \item 则在 \([a, b]\) 上存在唯一的不动点 \(x^*\),即满足 \(x^* = \varphi(x^*)\) 的点。
    \item 迭代格式 \(x_{k+1} = \varphi(x_k)\) 对任意的初值 \(x_0 \in [a, b]\) 均收敛于 \(\varphi(x)\) 的不动点 \(x^*\)。
    \item 并有误差估计式
\(\displaystyle
|x^* - x_k| \leq \frac{L^k}{1 - L} |x_1 - x_0|
\)。
\end{enumerate}
\end{theorem}

\begin{proof}
(1) \textbf{存在性}:
令 \(\psi(x) = x - \varphi(x)\),则有
\[
\psi(a) = a - \varphi(a) \leq 0, \quad \psi(b) = b - \varphi(b) \geq 0.
\]
由介值定理,存在 \(x^*\), \(a \leq x^* \leq b\),使得
\[
\psi(x^*) = x^* - \varphi(x^*) = 0 \quad \text{或} \quad x^* = \varphi(x^*).
\]

\textbf{唯一性}(反证法):
另一方面,若另有 \(x^+\) 满足 \(x^+ = \varphi(x^+)\),则由
\[
|x^* - x^+| = |\varphi(x^*) - \varphi(x^+)| = |\varphi'(\xi)(x^* - x^+)| \leq L|x^* - x^+|, \quad \xi \in [a,b]
\]
以及 \(L < 1\),得到 \(x^* = x^+\)。

(2) 当 \(x_0 \in [a,b]\) 时可用归纳法证明,迭代序列 \(\{x_k\} \subset [a,b]\),由微分中值定理
\begin{gather*}
    x_{k+1} - x^* = \varphi(x_k) - \varphi(x^*) = \overset{\xi \in [a,b]}{\varphi'(\xi)}(x_k - x^*)\leq L (x_k - x^*)
\\
|x_{k+1} - x^*| \leq L|x_k - x^*| = L|\varphi(x_{k-1}) - \varphi(x^*)| \leq L^2|x_{k-1} - x^*| \leq \cdots \leq L^{k+1}|x_0 - x^*|
\end{gather*}
因为 \(L < 1\),所以当 \(k \to \infty\) 时,\(L^{k+1} \to 0\),\(x_{k+1} \to x^*\),迭代格式 \(x_{k+1} = \varphi(x_k)\) 收敛。

(3) 误差估计:
\[
|x_{k+1} - x_k| = |\varphi(x_k) - \varphi(x_{k-1})| \leq L|x_k - x_{k-1}| \leq \cdots \leq L^k|x_1 - x_0|.
\]
设 \(k\) 固定,对于任意的正整数 \(p\) 有
\begin{align*}
|x_{k+p} - x_k| &\leq |x_{k+p} - x_{k+p-1}| + |x_{k+p-1} - x_{k+p-2}| + \cdots + |x_{k+1} - x_k| \\
&\leq (L^{k+p-1} + L^{k+p-2} + \cdots + L^k)|x_1 - x_0| = \frac{L^k(1 - L^p)}{1 - L}|x_1 - x_0|.
\end{align*}
由于 \(p\) 的任意性及
\[
\lim_{p \to \infty} x_{k+p} = x^*,
\]
故有 \(x^* = \varphi(x^*)\),
\[
|x^* - x_k| \leq \frac{L^k}{1 - L}|x_1 - x_0|.
\]
\end{proof}

构造收敛迭代格式的两个要素
\begin{enumerate}
    \item 选取合适的等价形式 \(x = \varphi(x)\),使得 \(\varphi(x)\) 满足上述定理的条件。
    \item 选取合适的初始值 \(x_0\)。必需取自 \(x^*\) 的充分小邻域,这个邻域大小决定于函数 \(f(x)\),以及做出的等价形式 \(x = \varphi(x)\)。
\end{enumerate}
\subsection{Newton迭代法}
Newton迭代格式为
\[x_{k+1}=x_k -\frac{f(x_k)}{f'(x_k)},\quad k=0,1,2,\cdots\]

Newton迭代对应\(f(x)=0\)的迭代方程是\(\varphi(x)=x-\frac{f(x)}{f'(x)}\),\(\displaystyle\varphi'(x)=\frac{f(x)f''(x)}{(f'(x))^2}\)。
\begin{enumerate}
    \item 若\(\alpha\)是\(f(x)\)的单根时,\(f(\alpha)=0\text{,}f'(\alpha)\neq 0\),则有\(|\varphi'(x_0)|=0\),
    只要初值\(x_0\)足够接近\(\alpha\),则迭代收敛且收敛阶为2。
    \item 若\(\alpha\)是\(f(x)\)的\(p\)重根时设 \(\alpha\) 为 \(f(x)\) 的 \(p\) 重根时,记
\begin{gather*}
f(x) = (x - \alpha)^p h(x) \\
\varphi(x) = x - \frac{(x - \alpha)^p h(x)}{p(x - \alpha)^{p-1}h(x) + (x - \alpha)^p h'(x)} = x - \frac{(x - \alpha)h(x)}{ph(x) + (x - \alpha)h'(x)} \\
\varphi'(x) = \frac{\left( 1 - \frac{1}{p} \right) + (x - \alpha)\frac{2h'(x)}{ph(x)} + (x - \alpha)^2\frac{h''(x)}{p^2h(x)}}
{\left[ 1 + (x - \alpha)\frac{h'(x)}{ph(x)} \right]^2 }\\
\varphi'(\alpha) = 1 - \frac{1}{p}
\end{gather*}
仍然有 \(|\varphi'(\alpha)| < 1\),当初始值在根 \(\alpha\) 附近,迭代也收敛,这是一阶迭代方法.

若 \(\alpha\) 为 \(f(x)\) 的 \(p\) 重根时,这时取下面迭代格式,仍是二阶方法
\begin{enumerate}
    \item \[
x_{k+1} = x_k - p\frac{f(x_k)}{f'(x_k)}, \quad k = 1, 2, \ldots
\]
    \item 取\(g(x) = \frac{f(x)}{f'(x)}\),则迭代格式为
    \begin{align*}
        x_{k+1} &= x_k - \frac{g(x_k)}{g'(x_k)}, \quad k = 1, 2, \ldots\\
    &=x_k-\frac{f(x_k)f'(x_k)}{(f'(x_k))^2 - f(x_k)f''(x_k)}
    \end{align*}
    \begin{note}
        如果 \(\alpha\) 是\(f(x) \)的\(p\)重根,那么\(\alpha\)是\(g(x)\) 的单根。因此,对函数 \(g(x)\) 进行 Newton 迭代,能够恢复二阶收敛性。
    \end{note}
\end{enumerate}

\end{enumerate}
\subsection{弦截法}
在Newton迭代格式中,用差商代替导数,并给定两个初值\(x_0\)和\(x_1\),得到弦截法迭代格式:
\[x_{k+1}=x_k -\frac{f(x_k)(x_k - x_{k-1})}{f(x_k)-f(x_{k-1})},\quad k=1,2,\cdots\]
弦截法的收敛阶为\(\frac{1+\sqrt{5}}{2}\approx 1.618\),介于线性收敛和二次收敛之间。
\subsection{对于非线性方程组的Newton方法}
对于非线性方程组:
\[
\vvec{F}(\vvec{x}) = \vvec{0}, \quad \vvec{F}: \mathbb{R}^n \to \mathbb{R}^n
\]

Newton迭代法的基本格式为:
\(
\vvec{x}^{(k+1)} = \vvec{x}^{(k)} - [J_F(\vvec{x}^{(k)})]^{-1} \vvec{F}(\vvec{x}^{(k)})
\),其中 $J_F(\vvec{x})$ 是 Jacobi 矩阵:
\[
J_F(\vvec{x}) = \begin{bmatrix}
\frac{\partial f_1}{\partial x_1} & \frac{\partial f_1}{\partial x_2} & \cdots & \frac{\partial f_1}{\partial x_n} \\
\frac{\partial f_2}{\partial x_1} & \frac{\partial f_2}{\partial x_2} & \cdots & \frac{\partial f_2}{\partial x_n} \\
\vdots & \vdots & \ddots & \vdots \\
\frac{\partial f_n}{\partial x_1} & \frac{\partial f_n}{\partial x_2} & \cdots & \frac{\partial f_n}{\partial x_n}
\end{bmatrix}
\]

为避免直接求逆矩阵,实际计算采用解线性方程组的形式:
\begin{align*}
J_F(\vvec{x}^{(k)}) \Delta \vvec{x}^{(k)} &= -\vvec{F}(\vvec{x}^{(k)}) \\
\vvec{x}^{(k+1)} &= \vvec{x}^{(k)} + \Delta \vvec{x}^{(k)}
\end{align*}

newton\_system\_solver.m是我编写的 MATLAB 代码,实现了用于求解非线性方程组的 Newton 迭代法:
% \begin{multicols}{2}
    \begin{lstlisting}[caption={newton\_system\_solver.m}]
        function [solution, iterations, errors] = newton_system(f, vars, x0, max_iter, tol)
        % 使用 Newton 迭代法求解非线性方程组
        % 
        % 输入:
        %   f - 符号函数向量
        %   vars - 符号变量向量  
        %   x0 - 初始猜测
        %   max_iter - 最大迭代次数 (默认: 50)
        %   tol - 容差 (默认: 1e-12)
        % 
        % 输出:
        %   solution - 数值解
        %   iterations - 实际迭代次数
        %   errors - 每次迭代的误差历史
        
        if nargin < 4
            max_iter = 50;
        end
        if nargin < 5
            tol = 1e-12;
        end
        
        % 计算 Jacobian 矩阵
        J = jacobian(f, vars);
        
        % 转换为数值函数句柄
        f_func = matlabFunction(f, 'Vars', {vars});
        J_func = matlabFunction(J, 'Vars', {vars});
        
        x = x0;
        errors = zeros(max_iter, 1);
        
        for iterations = 1:max_iter
            % 计算当前函数值和 Jacobian
            f_val = f_func(x);
            J_val = J_func(x);
            
            % 计算误差
            current_error = norm(f_val);
            errors(iterations) = current_error;
            
            % 检查收敛
            if current_error < tol
                break;
            end
            
            % Newton 迭代
            delta_x = -J_val \ f_val;
            x = x + delta_x;
            
            % 检查步长是否太小
            if norm(delta_x) < 1e-14
                disp('delta_x 小于1e-14')
                break;
            end
        end
    
        solution = x;
    end
    \end{lstlisting}
% \end{multicols}

我让DeepSeek帮我优化了代码,提高了运行效率,见newton\_system\_solver.m。
\section{求解线性方程组的迭代方法}
考虑线性方程组 $Ax = b$,将矩阵 $A$ 分解为 $A = N - P$,其中 $N$ 可逆,得到等价形式:\(x = N^{-1}Px + N^{-1}b\)。
令 $M = N^{-1}P$,$g = N^{-1}b$,则原方程等价于:
\(x = Mx + g\)

构造迭代格式:
\[
x^{(k)} = Mx^{(k-1)} + g, \quad k = 1, 2, \dots
\]

设迭代序列 $\{x^{(k)}\}$ 收敛于 $x^*$,则对迭代式两边取极限得:
\[
x^* = Mx^* + g,
\]
即 $x^*$ 是 $Ax = b$ 的解。

误差向量 $e^{(k)} = x^* - x^{(k)}$ 满足:
\begin{gather*}
e^{(k)} = M e^{(k-1)} = M^2 e^{(k-2)} = \cdots = M^k e^{(0)}, \\
\|e^{(k)}\| = \|M^k e^{(0)}\| \leq \|M^k\| \cdot \|e^{(0)}\|.
\end{gather*}
若 $\lim_{k \to \infty} \|M^k\| = 0$,则 $\lim_{k \to \infty} e^{(k)} = 0$,迭代收敛。

由线性代数定理知:
\[
\lim_{k \to \infty} M^k = O \quad \Leftrightarrow \quad \rho(M) < 1,
\]
其中 $\rho(M)$ 为 $M$ 的谱半径。又因为 $\|M\| \geq \rho(M)$,所以若 $\|M\| < 1$,则 $\rho(M) < 1$,迭代收敛。
\subsection{简单Jacobi迭代}

设线性方程组 $Ax = b$,其中 $A = (a_{ij}) \in \mathbb{R}^{n \times n}$ ,且 $a_{ii} \neq 0, (i = 1, 2, \dots, n)$。

将 $A$ 分解为:
\[
A = D + A - D
\]
其中:
\begin{itemize}
    \item $D = \text{diag}(a_{11}, a_{22}, \dots, a_{nn})$ 为对角矩阵,设\(D\)可逆。
\end{itemize}

则\begin{gather*}
    Ax = (D + A - D)x = b \Leftrightarrow Dx = (D - A)x + b 
\end{gather*}

Jacobi 迭代法的迭代格式为:
\begin{enumerate}
    \item 矩阵形式为:
\begin{gather*}
x^{(k)} = D^{-1}(D - A)x^{(k-1)} + D^{-1}b, \quad k = 1, 2, \dots\\
\text{记}\quad R=I-D^{-1}A,\quad g=D^{-1}b\\
x^{(k)} = Rx^{(k-1)} + g\\
\text{即}\quad \begin{pmatrix}
x_{1}^{(k)} \\
x_{2}^{(k)} \\
\vdots \\
x_{n}^{(k)}
\end{pmatrix}
=
\begin{pmatrix}
0 & r_{12} & \cdots & r_{1n} \\
r_{21} & 0 & \cdots & r_{2n} \\
\vdots & \vdots & \ddots & \vdots \\
r_{n1} & r_{n2} & \cdots & 0
\end{pmatrix}
\begin{pmatrix}
x_{1}^{(k-1)} \\
x_{2}^{(k-1)} \\
\vdots \\
x_{n}^{(k-1)}
\end{pmatrix}
+
\begin{pmatrix}
g_{1} \\
g_{2} \\
\vdots \\
g_{n}
\end{pmatrix}\\
r_{ij} = -\frac{a_{ij}}{a_{ii}}, \quad g_i = \frac{b_i}{a_{ii}}, \quad r_{ii} = 0
\end{gather*}

\item 分量形式为:
\begin{gather*}
x_i^{(k)} = \frac{1}{a_{ii}} \left( b_i - \sum_{j=1}^{i-1} a_{ij}x_j^{(k-1)} - \sum_{j=i+1}^{n} a_{ij}x_j^{(k-1)} \right), \quad i = 1, 2, \dots, n
\end{gather*}
\end{enumerate}

迭代收敛的充分必要条件是:
\[
\rho(R) < 1
\]
其中 $\rho(R)$ 是 Jacobi 迭代矩阵的谱半径。
\begin{theorem}
    若 $M$ 严格对角占优,则 $M$ 可逆。
\begin{proof}
     设 $M$ 严格行对角占优。若 $M$ 不可逆,则存在非零向量 $x = (x_1, \dots, x_n)^T$ 使 $Mx = 0$。
    取 $|x_i| = \max(|x_1|, \dots, |x_n|)$,则:
    \[
    |m_{ii}x_i| = \left|\sum_{j\neq i} m_{ij}x_j\right| \leq \sum_{j\neq i} |m_{ij}|\cdot|x_i| < |m_{ii}|\cdot|x_i|
    \]
    矛盾。若 $M$ 严格列对角占优,则 $M^T$ 严格行对角占优。综上,$M$ 可逆。
\end{proof}
\end{theorem}
\begin{theorem}
    若 $A$ 严格对角占优,则 Jacobi 迭代收敛。
\begin{proof}
\begin{enumerate}
    \item[] 若 $A$ 严格行对角占优:
    \begin{gather*}
    \rho(I - D^{-1}A) \leq \|I - D^{-1}A\|_\infty 
    = \max_{1 \leq i \leq n} \sum_{j \neq i} \frac{|a_{ij}|}{|a_{ii}|} < 1
    \end{gather*}
    \item[] 若 $A$ 严格列对角占优:
    \begin{gather*}
    \rho(I - D^{-1}A) = \rho(I - AD^{-1}) \leq \|I - AD^{-1}\|_1 
    = \max_{1 \leq j \leq n} \sum_{i \neq j} \frac{|a_{ij}|}{|a_{jj}|} < 1
    \end{gather*}
\end{enumerate}
\begin{note}
    \begin{gather*}
        \det(D)\det(\lambda I - I + D^{-1}A)\det(D^{-1}) = \det(\lambda I - I + AD^{-1})\\
        \Leftrightarrow \text{特征值一致} \Rightarrow \rho(I - D^{-1}A) = \rho(I - AD^{-1})
    \end{gather*}
\end{note}
\end{proof}
\end{theorem}

\subsection{Gauss-Seidel迭代}

设线性方程组 $Ax = b$,其中 $A = (a_{ij}) \in \mathbb{R}^{n \times n}$,且 $a_{ii} \neq 0, (i = 1, 2, \dots, n)$。

将 $A$ 分解为:
\[
A = D + L + U
\]
其中:
\begin{itemize}
    \item $D = \text{diag}(a_{11}, a_{22}, \dots, a_{nn})$ 为对角矩阵
    \item $L$ 为严格下三角部分($l_{ij} = a_{ij}$ for $i > j$, $l_{ij} = 0$ for $i \leq j$)
    \item $U$ 为严格上三角部分($u_{ij} = a_{ij}$ for $i < j$, $u_{ij} = 0$ for $i \geq j$)
\end{itemize}

则\begin{gather*}
    Ax = (D + L + U)x = b \Leftrightarrow (D + L)x = -Ux + b 
\end{gather*}

Gauss-Seidel 迭代法的迭代格式为:
\begin{enumerate}
    \item 矩阵形式为:
\begin{gather*}
x^{(k)} = -(D + L)^{-1}Ux^{(k-1)} + (D + L)^{-1}b, \quad k = 1, 2, \dots\\
\text{记}\quad G = -(D + L)^{-1}U,\quad g = (D + L)^{-1}b\\
x^{(k)} = Gx^{(k-1)} + g
\end{gather*}

\item 分量形式为:
\begin{gather*}
x_i^{(k)} = \frac{1}{a_{ii}} \left( b_i - \sum_{j=1}^{i-1} a_{ij}x_j^{(k)} - \sum_{j=i+1}^{n} a_{ij}x_j^{(k-1)} \right), \quad i = 1, 2, \dots, n
\end{gather*}
\end{enumerate}

迭代收敛的充分必要条件是:
\[
\rho(G) < 1
\]
其中 $\rho(G)$ 是 Gauss-Seidel 迭代矩阵的谱半径。

\begin{theorem}
    若 $A$ 严格对角占优,则 Gauss-Seidel 迭代收敛。
\begin{proof}
设 $A$ 严格行对角占优。记误差向量 $e^{(k)} = x^* - x^{(k)}$,其中 $x^*$ 为精确解。

由 Gauss-Seidel 迭代格式:
\begin{gather*}
e^{(k)} = -(D + L)^{-1}U e^{(k-1)}
\end{gather*}

设 $\|e^{(k)}\|_\infty = |e_i^{(k)}|$,则:
\begin{gather*}
|e_i^{(k)}| = \left| -\frac{1}{a_{ii}} \left( \sum_{j=1}^{i-1} a_{ij}e_j^{(k)} + \sum_{j=i+1}^{n} a_{ij}e_j^{(k-1)} \right) \right| \\
\leq \frac{1}{|a_{ii}|} \left( \sum_{j=1}^{i-1} |a_{ij}|\cdot|e_j^{(k)}| + \sum_{j=i+1}^{n} |a_{ij}|\cdot|e_j^{(k-1)}| \right) \\
\leq \frac{1}{|a_{ii}|} \left( \sum_{j=1}^{i-1} |a_{ij}|\cdot\|e^{(k)}\|_\infty + \sum_{j=i+1}^{n} |a_{ij}|\cdot\|e^{(k-1)}\|_\infty \right)
\end{gather*}

整理得:
\begin{gather*}
\|e^{(k)}\|_\infty \leq \frac{\sum_{j=i+1}^{n} |a_{ij}|}{|a_{ii}| - \sum_{j=1}^{i-1} |a_{ij}|} \|e^{(k-1)}\|_\infty
\end{gather*}

由于 $A$ 严格对角占优,有:
\begin{gather*}
\frac{\sum_{j=i+1}^{n} |a_{ij}|}{|a_{ii}| - \sum_{j=1}^{i-1} |a_{ij}|} < 1
\end{gather*}

故 $\|e^{(k)}\|_\infty \to 0$,Gauss-Seidel 迭代收敛。
\end{proof}
\end{theorem}

\begin{theorem}
    若 $A$ 对称正定,则 Gauss-Seidel 迭代收敛。
\begin{proof}
由于 $A$ 对称正定,$D$ 正定,$L$ 和 $U$ 满足 $U = L^T$。

设 $\lambda$ 为 $G$ 的特征值,$x$ 为对应特征向量,则:
\begin{gather*}
Gx = \lambda x \Rightarrow -(D + L)^{-1}Ux = \lambda x \Rightarrow -Ux = \lambda(D + L)x
\end{gather*}

考虑二次型:
\begin{gather*}
x^H Ax = x^H (D + L + U)x = x^H (D + L - \lambda(D + L))x \\
= (1 - \lambda)x^H (D + L)x
\end{gather*}

由于 $A$ 正定,$x^H Ax > 0$,且 $D + L$ 正定,故 $|\lambda| < 1$,即 $\rho(G) < 1$。
\end{proof}
\end{theorem}









%  ↑↑↑↑↑↑↑↑↑↑↑↑↑↑↑↑↑↑↑↑↑↑↑↑↑↑↑↑ 正文部分
\ifx\allfiles\undefined
\end{document}
\fi