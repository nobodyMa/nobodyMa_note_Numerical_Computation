\ifx\allfiles\undefined
\documentlecture[12pt, a4paper, oneside, UTF8]{ctexbook}  %  这一句是新增加的
\usepackage[dvipsnames]{xcolor}
\usepackage{amsmath}   % 数学公式
\usepackage{graphicx}
% \usetikzlibrary{arrows, calc, decorations.pathmorphing}
\allowdisplaybreaks % 允许公式跨页换行
\newcommand{\pa}{\partial}
\newcommand{\mT}{\raisebox{0.1ex}{$\scriptstyle -T$}} % 调整高度为 0.1ex
\newcommand{\lmT}{\raisebox{-0.85ex}{$\scriptstyle -T$}} % 调整高度为 -0.85ex
\newcommand{\mone}{\raisebox{0.1ex}{$\scriptstyle -1$}} % 调整高度为 0.1ex
\newcommand{\lmone}{\raisebox{-0.85ex}{$\scriptstyle -1$}} % 调整高度为 -0.85ex
\newcommand{\lsup}[1]{\raisebox{-0.1ex}{$\scriptstyle #1$}}
\newcommand{\llsup}[1]{\raisebox{-0.85ex}{$\scriptstyle #1$}}
\newcommand{\mathminus}{\!\!-\!\!} % 数学环境连字符
\newcommand{\vvec}[1]{\symbf{#1}}
\newcommand{\ten}[1]{\mathbb{#1}}
\newcommand{\red}[1]{\textcolor{red}{#1}}

\begin{document}
%\title{\Huge{\textbf{}}}
\author{作者:无名氏马}
\date{\today}
\maketitle                   % 在单独的标题页上生成一个标题

\thispagestyle{empty}        % 前言页面不使用页码
\begin{center}
    \Huge\textbf{前言}
\end{center}


\begin{flushright}
    \begin{tabular}{c}
        \today
    \end{tabular}
\end{flushright}

\newpage                      % 新的一页
\pagestyle{plain}             % 设置页眉和页脚的排版方式(plain:页眉是空的,页脚只包含一个居中的页码)
\setcounter{page}{1}          % 重新定义页码从第一页开始
\pagenumbering{Roman}         % 使用大写的罗马数字作为页码
\tableofcontents              % 生成目录

\newpage                      % 以下是正文
\pagestyle{plain}
\setcounter{page}{1}          % 使用阿拉伯数字作为页码
\pagenumbering{arabic}
% \setcounter{chapter}{-1}    % 设置 -1 可作为第零章绪论从第零章开始
 % 单独编译时,其实不用编译封面目录之类的,如需要不注释这句即可
\else
\fi
%  ↓↓↓↓↓↓↓↓↓↓↓↓↓↓↓↓↓↓↓↓↓↓↓↓↓↓↓↓ 正文部分
\chapter{数值计算}
\section{绪论}
\begin{definition}
    若\(A\)非奇异,称\(cond(A)=\|A\|\|A^{-1}\|\)为\(A\)的条件数。
    \(cond(A)\ge 1\),当\(A\)为正交矩阵时\(cong(A)=1\)。
\end{definition}
\begin{theorem}
    对于线性方程组\(Ax=b\)
    \begin{enumerate}
        \item 设\(A(x+\delta x)=b+\delta b\text{,}\delta x\text{收到}\delta b \text{的影响表示为}\)
        \[\frac{\|\delta x\|}{\|x\|}\le cond(A)\frac{\|\delta b\|}{\|b\|}\]
        \item 设\((A+\delta A)(x+\delta x)=b+\delta b\text{,}\delta x\text{收到}\delta A,\delta b \text{的影响表示为}\)
        \[\frac{\|\delta x\|}{\|x\|}\le \frac{cond(A)}{1-cond\frac{\|\delta A\|}{\|A\|}}\left(\frac{\|\delta A\|}{\|A\|}
        +\frac{\|\delta b\|}{\|b\|}\right)\]
    \end{enumerate}
    \begin{proof}
        \begin{gather*}
            (A+\delta A)(x+\delta x)=b+\delta b\\
            (A+\delta A)\delta x=b+\delta b -\cancelto{=b+\delta A\cdot x}{A+\delta A}\\
            \delta x=\underbrace{(A+\delta A)^{-1}}_{(I+\delta A A^{-1})^{-1}}(\delta b -\delta A\cdot x)\\
            \|\delta x\|=\|A^{-1}(I+\delta A A^{-1})^{-1}(\delta b-\delta A \cdot x)\|
            \\\le \|A^{-1}\|\|(I+\delta A A^{-1})^{-1}\|\|(\delta b-\delta A \cdot x)\|\\
            \le  \|A^{-1}\|\frac{1}{1-\|\delta A\|\| A^{-1}\|}\|(\underbrace{\delta b}_{partA}-\underbrace{\delta A \cdot x}_{partB})\|\\
            partB = \|A^{-1}\|\|A\|\frac{1}{1-\|\delta A\|\| A^{-1}\|}\frac{\|x\|\|\delta A\|}{\|A\|}\\
            partA=\|A^{-1}\|\frac{1}{1-\|\delta A\|\| A^{-1}\|}\frac{\|\delta b\|\|Ax\|}{\|B\|}
            \le \|A^{-1}\|\frac{1}{1-\|\delta A\|\| A^{-1}\|}\frac{\|\delta b\|\|A\|\|x\|}{\|B\|}\\
            \frac{\|\delta x\|}{\|x\|}\le \frac{cond(A)}{1-cond\frac{\|\delta A\|}{\|A\|}}\left(\frac{\|\delta A\|}{\|A\|}
        +\frac{\|\delta b\|}{\|b\|}\right)
        \end{gather*}
    \end{proof}
\end{theorem}
\begin{lemma}
    若\(A\in \R^{n\times n}\text{,且}\|A\|<1\),
    则\begin{enumerate}
        \item \(I-A\)非奇异
        \item \((I-A)^{-1}=\sum_{k=0}^{\infty}A^k\)
        \item \(\|(I-A)^{-1}\|\le \frac{1}{1-\|A\|}\)
    \end{enumerate}
    \begin{proof}
        \begin{enumerate}
            \item 
        若\(I-A\)奇异,则\(\exists x\neq0\text{,}s.t.(I-A)x=0\text{,即1为}A\text{的特征值,矛盾}\)
        \item 
        \(\sum_{k=0}^{\infty}A^k\cdot (I-A)=A^0=I\)
        \item         \(\|(I-A)^{-1}\|= \|\sum_{k=0}^{\infty}A^k\|\le \sum_{k=0}^{\infty}\|A^k\|
        \le \sum_{k=0}^{\infty}\|A\|^k=\frac{1}{1-\|A\|}\)
    \end{enumerate}
    \end{proof}
\end{lemma}
\begin{note}
在数值分析中,判断矩阵 $A$ 是否病态(即对微小扰动敏感)通常不直接计算 $A^{-1}$(条件数),而是依据以下经验准则:
\begin{enumerate}
    \item \textbf{行列式很大或很小(如某些行、列近似相关)} \\
    行列式 $\det(A)$ 的绝对值若远小于或远大于矩阵元素数量级的 $n$ 次方,可能预示病态。\\
    \textbf{解释:} 行列式绝对值过小通常意味着矩阵的行或列向量近似线性相关,导致其张成的“体积”很小,矩阵接近奇异。

    \item \textbf{元素间相差大数量级,且无规则} \\
    若矩阵元素在数值上跨越多个数量级(如 $10^{-6}$ 与 $10^3$ 共存),且没有特定的分布模式(如对角占优)。\\
    \textbf{解释:} 这种巨大的尺度差异会在数值计算(如消元法)中放大舍入误差,除非该矩阵具有特殊的、可补偿的结构。

    \item \textbf{主元消去过程中出现小主元} \\
    在进行高斯消元时,如果出现绝对值非常小的主元。\\
    \textbf{解释:} 小主元意味着矩阵接近奇异,在消元过程中需要用其去除其他大元素,这会极大地放大原始数据中存在的微小误差。

    \item \textbf{特征值相差大数量级} \\
    矩阵的特征值 $\lambda_i$ 的绝对值分布范围极广,即 $\max|\lambda_i| / \min|\lambda_i| \gg 1$。\\
    \textbf{解释:} 对于对称矩阵,这直接导致其条件数 $\kappa(A)$ 极大,意味着矩阵在不同特征向量方向上的伸缩程度差异巨大,从而对扰动非常敏感。
\end{enumerate}
\end{note}

\section{非线性方程求根}
\subsection{二分法(对分法)}
\mbox{}
二分法两种终止条件:
\begin{enumerate}
    \item  给定精度条件 \( |f(x)| < \varepsilon \)
    \textbf{不适用情况:}  
    当函数在远离根的区域存在\textbf{平坦区间}(即导数近似为零)时,即使当前点 \(x\) 离真实根 \(r\) 还很远,也可能满足 \( |f(x)| < \varepsilon \),导致算法提前终止,返回一个错误解。
    
    数学描述:  
    存在连续函数 \( f \in C[a,b] \),\( f(r) = 0 \),且存在某点 \( x_0 \) 满足 \( |x_0 - r| \gg 0 \),但 \( |f(x_0)| < \varepsilon \)。  
    此时若以 \( |f(x_n)| < \varepsilon \) 为终止条件,可能在 \( x_n = x_0 \) 时就停止迭代,而实际误差 \( |x_n - r| \) 很大。
    
    \item 区间长度条件 \( b - a < \varepsilon \)
    \textbf{不适用情况:}  
    当函数在根 \( r \) 附近\textbf{数值上呈现陡峭跳变}(由于浮点舍入误差导致类似符号函数的行为)时,可能在迭代过程中因符号判断错误而将真根排除在区间之外。此时区间继续缩小至 \( b-a < \varepsilon \),但区间内不含根,导致返回错误解。
    
    数学描述:  
    存在连续函数 \( f \in C[a,b] \),由于浮点运算精度限制,在根 \( r \) 的某邻域 \( (r - \delta, r + \delta) \) 之外,计算出的 \(\mathrm{sign}(f(x))\) 为常值且错误,导致二分法过早丢弃含根区间。此后区间 \([a_n, b_n]\) 满足 \( b_n - a_n < \varepsilon \) 但不满足 \( r \in [a_n, b_n] \)。
\end{enumerate}







%  ↑↑↑↑↑↑↑↑↑↑↑↑↑↑↑↑↑↑↑↑↑↑↑↑↑↑↑↑ 正文部分
\ifx\allfiles\undefined
\end{document}
\fi